\begin{figure}[t]
    \centering
    \begin{adjustbox}{height=0.58\textheight} % Scale a `tikzpicture`, see https://tex.stackexchange.com/a/11536/61591
        \begin{tikzpicture}[yscale=-1] % See https://tex.stackexchange.com/a/90012/61591
            \draw[thick] (0,0) grid (32,32); % See https://tex.stackexchange.com/a/552215
            \draw[thick,->] (0,0) -- (33,0) node[anchor=south west] {\(x\)}; % See https://www.overleaf.com/learn/latex/LaTeX_Graphics_using_TikZ%3A_A_Tutorial_for_Beginners_(Part_1)%E2%80%94Basic_Drawing
            \draw[thick,->] (0,0) -- (0,33) node[anchor=north east] {\(y\)};
            \foreach \x in {1,...,32}
            \draw (\x cm,1pt) -- (\x cm,-1pt) node[anchor=south] {\(\x\)};
            \foreach \y in {0,...,32}
            \draw (1pt,\y cm) -- (-1pt,\y cm) node[anchor=east] {\(\y\)};
            % See https://tex.stackexchange.com/a/21164/61591
            \node[circle,fill,draw,minimum size=1.5cm,inner sep=0pt,text=white] (charge1) at (8,4) {\((8, 4)\)};
            \node[circle,fill,draw,minimum size=1.5cm,inner sep=0pt,text=white] (charge2) at (24,4) {\((24, 4)\)};
            \newcommand\Square[1]{+(-#1,-#1) rectangle +(#1,#1)} % See https://tex.stackexchange.com/a/58930/61591
            \fill (20,24) \Square{4};
            \node[text=white] at (20,24) {\Large \((20, 24)\)};
        \end{tikzpicture}
    \end{adjustbox}
    \caption{A schematic plot of the two-dimensional region of size \(L = 32\)
        (i.e., \(N = 33\)).
        Note that the positive direction of the \(y\)-axis is pointing downward.
        The circles are \(2\) point charges, and the square is held at a constant potential
        \(\phi = 5\).
        The tuples in the circles and the square are their centers' coordinates on
        this specific grid.}
    \label{fig:grid}
\end{figure}
